% By: Adam Clark
% Date: December 2018
% Prof: Benedict
% Class: CHY 113
% Title: Adam's Chem Project

\documentclass[14pt]{extarticle}

% Imports
\usepackage[utf8]{inputenc}
\usepackage{cancel}
\usepackage{amsmath}
\usepackage{graphicx}
\usepackage[urlcolor=blue]{hyperref}
\hypersetup{colorlinks=true}

% Macros
\newcommand{\chemElement}[4]{ % Make a tile with a chemical element on it
    \\~\\
    \fbox{
        \parbox{90}{
            \normalsize \flushleft
            {#3}\\
            \center \huge
            {#1}
            \center \normalsize
            {#2}\\
            
            \small
            {#4}
            \\~\\
        }
    }
} % \chemElement{H}{Hydrogen}{1}{1.008}



% Begin Doc
\begin{document}
    %: Title Page
    \center
    \LARGE Chem Project \\ 
    \normalsize
    by Adam Clark\\
    December 6 2018
    \vspace{\fill}
    
    \flushleft
    Abstract: We will focus on Dimensional Analysis and Molarity.
    \pagebreak
    
    %: 0th Concept: Dimensional analysis
    \flushleft
    \large DIMENSIONAL ANALYSIS \\
    \normalsize
    Dimensional analysis is a way to convert units. 
    \\~~~\\
    Part 1: Simple Unit Conversions\\
    Let's say we want to convert meters to centimeters. First we need to ask how many meters are in a centimeter. 
    
    \center
    $ 1 \text{Meter} = 100 \text{Centimeter} $
    \\ or \\
    $ {1 \over 100} \text{Meter} = 1 \text{Centimeter} $
    \flushleft
    
    Lets say we have 5 meters. We intuitively know that's 500 centimeters. How'd we do that?
    \\~~~\\
    Because 1 Meter is the same as 100 Centimeters we can say:

    \center \large
    ${100 \text{Centimeters} \over 1 \text {Meter}} = 1$
    \\~\\
    \flushleft \normalsize
    Using the definition intuitively we get:
    \center \large
    ${5 \text{Meters} \over 1} \times {100 \text{Centimeters} \over 1 \text {Meter}} = {500 \text{Centimeters} \over 1}$
    \\~\\
    \flushleft \normalsize
    The two 'Meters' cancel out. Leaving our units with Centimeters. 
    \center \large
    ${5 \cancel{\text{Meters}} \over 1} \times {100 \text{Centimeters} \over 1 \cancel{\text {Meter}}} = {500 \text{Centimeters} \over 1}$
    
    
    
    
    
    %: DIMENSIONAL ANALYSIS Try it
    \pagebreak
    \normalsize \flushleft
    Let's try it out:\\
    Given:\\
    \center
    $ 1 \text{Meter} = 100 \text{Centimeter} $
    \\ or \\
    $ {1 \over 100} \text{Meter} = 1 \text{Centimeter} $
    \flushleft
    \normalsize \flushleft
    How many Meters is 10,000 Centimeters?\\
    \large
    \hspace{2 cm} 0. Re-State as a fraction
        ${10,000 \text{Centimeters} \over 1}$
    \\~\\\hspace{2 cm} 1. Find Conversion Factor
        ${1 \text{Meter} \over 100 \text{Centimeter}}$
    \\~\\\hspace{2 cm} 2. Multiply to cancel units \\
        \center
        ${10,000 \cancel{\text{Centimeters}} \over 1} \times {1 \text{Meter} \over 100 \cancel{\text{Centimeters}}} = 100 \text{Meters}$
        \\~\\ or \\~\\
        $ {10,000 \cancel{\text{Centimeters}} \over 1} \div ({1 \text{Meter} \over 100 \cancel{\text{Centimeters}}})^{-1} = {10,000 \cancel{\text{Centimeters}} \over 1} \div {100 \cancel{\text{Centimeter}} \over 1 \text{Meters}} = 100 \text{Meters} $
    
    
    
    \flushleft \normalsize
    \pagebreak
    %: Multi-layer DIMENSIONAL ANALYSIS
    \flushleft
    Multi-layer DIMENSIONAL ANALYSIS\\
    \normalsize
    Let's say we have the conversion factors
    \center
    $ 1 \text{ Meter} = 100 \text{ Centimeter} $\\~\\
    $ 1 \text{ Inch} = 2.54 \text{ Centimeter} $
    \\~\\ \flushleft
    How do we convert from Meters to Inches? We can use dimensional analysis.
    \\~\\
    \center \large
    $ {x \text{ Meters} \over 1} \times {100 \text{ Centimeters} \over 1 \text{ Meters}} = z \text{ Centimeters} $\\~\\
    $ {z \text{ Centimeters} \over 1} \times {1 \text{ Inches} \over 2.54 \text{ Centimeters}} = y \text{ Inches} $
    \\~\\
    $ {x \text{ Meters} \over 1} \times {100 \text{ Centimeters} \over 1 \text{ Meters}} \times {1 \text{ Inches} \over 2.54 \text{ Centimeters}} = y \text{ Inches} $
    
    \flushleft \normalsize \\~\\
    Notice how the units cancel out.
    \center \large
      $ {x \cancel{\text{ Meters}} \over 1} \times {100 \cancel{\text{ Centimeters}} \over 1 \cancel{\text{ Meters}}} \times {1 \text{ Inches} \over 2.54 \cancel{\text{ Centimeters}}} = y \text{ Inches} $
      
      
      
      
    \pagebreak \normalsize
    %: DIMENSIONAL ANALYSIS and Chem
    \flushleft
    DIMENSIONAL ANALYSIS and Chemistry
    \\~\\
    \center
    \chemElement{H}{Hydrogen}{1}{1.008}
    \flushleft
    \\~\\
    \\~\\
    Let's say we have 10g of Hydrogen, how many moles is that? 
    \\To get our conversion factor we look at the periodic table and take the molar mass.
    \\~\\
    \center \large
    ${10 \text{ g of Hydrogen} \over 1} \times {1 \text{ mole of Hydrogen} \over 1.008 \text{ g of Hydrogen}} = 9.92 \text{ g of Hydrogen}$
    
    
    
    \pagebreak \flushleft
    %: DIMENSIONAL ANALYSIS and Chem Try it
    \center
    \chemElement{Na}{Sodium}{11}{22.990}
    \flushleft
    \\~\\
    \\~\\
    Let's say we have 1 mole of sodium. How many grams is that?
    \center \large
    ${1 \text{ mole of Na} \over 1} \times {22.990 \text{ g of Na} \over 1 \text{ mole of Na}} = 22.99 \text{ g of Na}$
    
    
    
    
    \pagebreak \flushleft
    %: DIMENSIONAL ANALYSIS Website
    \center
    Use the website for practice problems.
    \includegraphics[width=\textwidth]{Dim.png}
    \center \small \underline{ \url{https://adamclark2.github.io/ChemProject/dimensional_analysis.html} }
    
    
    
    
    
    
    
    
    
    
    
    
    
    
    
    %: 1st Concept: Molarity
    \pagebreak
    \flushleft
    \large MOLARITY \\
    \normalsize
    Molarity is defined as:
    $$ \text{Molarity} = {\text{Number of Moles} \over  \text{Liters}}$$
    
    Molarity is concentration. Let's visualize that:
    \includegraphics[width=\textwidth]{example.png}
    \center \small \underline{\url{https://adamclark2.github.io/ChemProject/visual.html}}
    
    \flushleft
    \\~\\
    As we can see above the left side has more 'atoms' than the right, but they have the same liter's.
    
    \\~\\
    We can also use this definition to do math. Let's say we have a 5 Molar solution and we have 2 liters of it, how many Moles do we have? 
    
    \center
    ${5 \text{ Moles} \over 1 \text { Liters}} \times {2 \text{ liters} \over 1} = 10 \text{ Moles}$
    

\end{document}
